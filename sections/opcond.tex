\section{最优性条件}\label{optimality condition}
下面我们推导问题\eqref{original problem matrix form}的KKT条件. 这对我们最终设置算法的停机准则会有帮助. 问题\eqref{original problem matrix form}的拉格朗日函数为
\begin{equation*}\begin{aligned}
&\mcL_0(X,Z,\lambda_1,\lambda_2,\mu,\Phi,\Omega)\\
&=f(X,Z)-\langle\lambda_1,X\one-\rho\rangle-\langle\lambda_2,Z^T\one-\rho\rangle\\
&-\mu\trace(X)-\langle\Omega,Z\rangle-\langle\Phi,X-Z\rangle,
\end{aligned}
\end{equation*}
其中$\mu\in\mathbb{R},\lambda_1,\lambda_2\in\mathbb{R}^n,\Phi,\Omega\in\mathbb{R}^{n\times n}$为拉格朗日乘子. 
根据文\cite{Nocedal2006Numerical}, 我们有问题\eqref{original problem matrix form}的KKT条件: 
若$(X^*,Z^*)$为问题\eqref{original problem matrix form}的解, 则存在拉格朗日乘子$\mu^*,\lambda_1^*,\lambda_2^*,\Phi^*,0\le\Omega^*$, 使得
\begin{equation}\left\{\begin{array}{ll}
\left.\begin{array}{l}\nabla_X\mcL_0=2R+Z^*R-\lambda_1^*\one^T-\Phi^*-\mu I=0,\\\nabla_Z\mcL_0=X^*R-\one\left(\lambda_2^*\right)^T+\Phi-\Omega=0,\end{array}\right\} & \begin{array}{l}
\makebox{稳定性条件}\\\makebox{或对偶可行性条件},
\end{array}\\
\left.\begin{array}{l}
X^*\one=\rho,\trace(X^*)=0,\\
\left(Z^*\right)^T\one=\rho,Z^*\ge0,\\
\Omega^*\ge0,
\end{array}\right\} & \makebox{原始可行性条件},\\
\Omega^*\circ Z^*=0.& \makebox{互补松弛条件}.
\end{array}\right.
\label{KKT of penalty problem}
\end{equation}