\section{引言}\label{introduction}
\subsection{问题背景与陈述}
本文所要研究的问题是一类特殊的双线性最优运输问题. 最优运输是多个学科交叉的研究领域, 包括概率、分析以及优化等. 最优运输研究的主要目标是建立有效比较概率分布的几何工具, 可见文\cite{Gu2017}. 以法国数学家Gaspard Monge在二百多年前给出的问题为例: 当给定两个沙盘时 (每盘沙子可以代表一个概率分布), 可以通过很多方式将一个沙盘运输到另一个沙盘. 基于运输单个沙粒的局部花费, 每一种运输方法均对应一个全局花费. 最优运输的目的就是寻找总体花费最少的运输方案, 从而进一步建立面向概率分布的几何工具集. 
\par 最优运输问题有着悠久而丰富的研究历史. 最早可以追溯到上文所提到十八世纪的Gaspard Monge. 而俄国数学家Kantorovich在上世纪四十年代给出了一种更实际的松弛形式, 并在上世纪九十年代因为一系列重要的数学理论成果得到进一步推动, 其中包括法国数学家Brenier的重要工作. 尤其重要的是, 多位菲尔兹奖获得者在最优运输理论研究中做出过重要贡献, 如法国数学家C\'edric Villani、意大利数学家Alessio Figalli, 并且已经有多本重要专著, 见文\cite{Villani2003Topics,Villani2008Optimal,Santambrogio2015Optimal,Peyre2019Computational}. 在应用方面, 最优运输被广泛用于计算机科学领域, 尤其是在计算机图形学、计算机视觉、医学图像处理以及深度学习等方面取得了显著效果.
\par 现在我们以现代数学语言陈述Gaspard Monge的最优运输问题:
\begin{dfn}[Monge问题]\label{monge}
	令$P,Q\subset\mathbb{R}^d$. 令$f$为$P$上的概率密度函数, $g$为$Q$上的概率密度函数. $c:P\times Q\to[0,\infty)$为连续函数. Monge问题是要求满足$T\#f=g$的运输映射$T:P\to Q$使得$T$最小化泛函
	$$M(T):=\int_Pc(p,T(p))f(p)\,\mathrm{d}p.$$
	这里$T\#f=g$表示$g$是$f$在$T$下的转移 (push-forward), 定义为
	$$\int_Bg(q)\,\mathrm{d}q=\int_{T^{-1}(B)}f(p)\,\mathrm{d}p,\quad\forall B\subset Q.$$换句话说, 在概率密度$g$下集合$B$的质量等于在概率密度$f$下集合$T^{-1}(B)$的质量.
\end{dfn}
注意定义\ref{monge}中的$c$可以看做是成本泛函. 当$f,g$为离散概率分布函数时, Monge问题就约减为一个线性规划 (LP). 假设$P,Q$是有限集, $P=\{p_1,\ldots,p_m\},Q=\{q_1,\ldots,q_n\}$. 对应地, 我们有它们的概率 (或质量)$\{f_1,\ldots,f_m\},\{g_1,\ldots,g_n\}$. 于是极小化定义\ref{monge}中的泛函等价于求解以下LP:
\begin{equation}
	\begin{array}{rl}
		\min\limits_{a_{ij}} & \sum\limits_{i,j}c_{ij}a_{ij}\\
		\st & \sum\limits_{j}a_{ij}=f_i,\quad i=1,\ldots,m,\\
		&\sum\limits_ia_{ij}=g_j,\quad j=1,\ldots,n,\\
		&a_{ij}\ge0,\quad i=1,\ldots,m,\,j=1,\ldots,n,
	\end{array}
	\label{monge LP}
\end{equation}
其中$c_{ij}:=c(x_i,y_j)$. 实际上此LP的输出$A=(a_{ij})_{m\times n}$就构成了一个运输映射. 其行指标代表源$P$中的位置, 列指标代表汇$Q$中的位置. 每个$a_{ij}$表示从$p_i$运输至$q_j$的总质量. 
\par 下面我们陈述本文考虑的一类非凸、双线性最优运输问题:
\begin{equation}
\begin{array}{ll}
\min\limits_{X,Y} & \sum\limits_{i\ne j}\frac{x_{ij}}{|r_i-r_j|}+\sum\limits_{i\ne k}\frac{y_{ik}}{|r_i-r_k|}+\sum\limits_{i,j,k:j\ne k}\frac{x_{ij}y_{ik}}{|r_j-r_k|}\\
\mathrm{s.t.} & \sum\limits_jx_{ij}=\rho_i,\quad i=1,2,\ldots,n,\\
& \sum\limits_ix_{ij}=\rho_j,\quad j=1,2,\ldots,n,\\
& \sum\limits_ky_{ik}=\rho_i,\quad i=1,2,\ldots,n,\\
& \sum\limits_iy_{ik}=\rho_k,\quad k=1,2,\ldots,n,\\
& x_{ij},y_{ik}\ge0,\quad i,j,k=1,2,\ldots,n,\\
& x_{ii},y_{ii}=0,\quad i=1,2,\ldots,n,
\end{array}
\label{original problem element form}
\end{equation}
其中$X=(x_{ij})$, $Y=(y_{ij})\in\mathbb{R}^{n\times n}$为待求的矩阵变量, $r=(r_1,r_2,\ldots,r_n)^T$, $\rho=(\rho_1,\rho_2,\ldots,\rho_n)^T\in\mathbb{R}^n_+$. 另外, 距离信息$\{|r_i-r_j|\}$储存于矩阵$R=(r_{ij})\in\mathbb{R}^{n\times n}$中, 其中$r_{ij}=1/|r_i-r_j|$. 我们由$R$的定义立可得其对称性.
另外, $\trace(R)=0$. 在此我们说明, 对应于$\rho_i$的等式约束与非负约束合在一起, 本质上是将第$i$行 (列)限制到单纯形上; $x_{ii},y_{ii}=0$的加入是为了排除平凡解, 即$X,Y$均是以$\{\rho_i\}$为对角元的对角矩阵.
\par 联系问题\eqref{original problem element form}与问题\eqref{monge LP}, 我们得出:
\begin{enumerate}[(i)]
\item 在问题\eqref{original problem element form}中, 有两个待求的运输映射$X=(x_{ij})_{n\times n}$和$Y=(y_{ij})_{n\times n}$. 换句话说, 有两个运输工具.
\item 在问题\eqref{original problem element form}中, 源集即是汇集.
\item 在问题\eqref{original problem element form}中, 成本泛函取成了距离的反比例函数.
\item 问题\eqref{original problem element form}中目标函数的前两项类似于问题\eqref{monge LP}的目标函数, 这表明运输映射$X,Y$应尽量少消耗成本, 并且每个位置的物质均要向其他位置运送, 不允许停留. 而问题\eqref{original problem element form}目标函数的最后一项则提出了更高的要求. 当$r_j$和$r_k$ (这里我们将它们看做欧式空间中的点)距离较近时, $X$倾向于不将$r_i$中的物质运输到$r_j$, 而$Y$则倾向于不将$r_i$中的物质运输到$r_k$. 但若$r_j=r_k$, 则这样带来的额外成本是0. 
\end{enumerate}
\subsection{研究现状}
问题\eqref{original problem element form}是一个双线性规划问题, 其一般形式为
\begin{equation}
	\begin{array}{rl}
		\min\limits_{x,y} & x^TQy+c^Tx+d^Ty\\
		\st & Ax+By\le b,\\
		 & 0\le x\le a_1,\\
		 & 0\le y\le a_2,
	\end{array}
	\label{bilinear programming}
\end{equation}
其中$b\in\mathbb{R}^p,a_1,c,x\in\mathbb{R}^{m},a_2,d,y\in\mathbb{R}^{n},Q\in\mathbb{R}^{m\times n},A\in\mathbb{R}^{p\times m},B\in\mathbb{R}^{p\times n}$. 更一般地, 问题\eqref{bilinear programming}从属于更大的一类问题——二次规划. 显然, 问题\eqref{bilinear programming}是一个非凸二次规划. 文\cite{Pardalos1991Quadratic}说明, 即使目标函数的二次项矩阵只有一个负特征值, 这个问题也是NP困难的. 
\par 双线性规划分为两大类, 一类为可分离约束的问题, 即问题\eqref{bilinear programming}的可行域可表示成
$$x\in \mathcal{X},y\in \mathcal{Y},$$这里$\mathcal{X},\mathcal{Y}$均为多面体. 另一类为联合约束的问题. 显然问题\eqref{original problem element form}是可分离约束的问题. 可分离约束双线性规划最早作为一个等价的二次规划, 出现在双矩阵博弈的求解中, 可见文\cite{Mills1960Equilibrium}. 对于此双矩阵博弈, 问题\eqref{bilinear programming}的目标函数有下界0, 且在问题全局解处达到此界. 为求解文\cite{Mills1960Equilibrium}中的问题, 文\cite{O1964Equilibrium}利用Rosen的梯度投影法 (见文\cite{Rosen1960The}), 而文\cite{MANGASARIAN1964348}则使用文\cite{Balinski1961An}的寻找多面体所有顶点的方法. 文\cite{Altman1968bilinear}考虑了无已知下界的问题, 并设计了求得局部解的算法.
\par 求解可分离约束双线性规划的主要方法包括割平面法、分支定界法、线性化方法及对偶技术. 割平面法均基于文\cite{Konno1976ACP}证明的关于可分离问题的解的性质: 如果多面体$\mathcal{X},\mathcal{Y}$非空且有界, 则可以找到问题的一个最优解$(x^*,y^*)$, 其中$x^*$是多面体$\mathcal{X}$的顶点, $y^*$是多面体$\mathcal{Y}$的顶点. 文\cite{Ritter1966}首先提出求解可分离问题的割平面法. 之后文\cite{Konno1976ACP}对其加以改进, 提出两阶段的算法. 尽管他的算法求解了所有的测试问题, 但他却不能保证算法能够收敛到全局解. 文\cite{Ding2007}基于文\cite{Konno1976ACP}的算法, 加入条件以提高下界上升速度, 提出了加速方法. 
\par 利用线性规划的对偶理论, 可分离约束问题可写作等价的极大极小问题. 对此构造, 文\cite{Falk1973}提出了一个有限终止的分支定界算法. 而文\cite{Gallo1977}则对文\cite{Tuy1964Concave}中针对凹极小化问题提出的两个算法加以改进. 其中第一个是无收敛保证的割平面法, 第二个是扩大多面体法. 当$\mathcal{X}$多面体由文\cite{Zwart1973Nonlinear}给出的反例中的约束定义时, 后者会陷入循环. 受文\cite{Gallo1977}启发, 文\cite{Vaish1976The,Vaish1977A}改进了文\cite{Tuy1964Concave}的策略, 利用极切得到了具有有限终止性的扩大多面体法和收敛的割平面法. 而后文\cite{Sherali1980}利用分离面切得到了割平面法的有限终止性. 文\cite{Audet1999}分析了一个具有无界定义域的双线性规划最优解是否有界的问题. 在他们的算法中, 变形后的双线性问题被嵌入了分支定界法中. 近期, 文\cite{Alarie2001COncavity}提出了结合凹切和文\cite{Audet1999}中的分支定界程序的方法. 这种方法首先用凹切减小可行域, 之后再约减的可行域上使用分支定界法. 不像传统的割平面法, 这种方法给$\mathcal{X},\mathcal{Y}$都加了凹切, 因此使得计算负荷更加沉重. 
\par 以上方法中, 文\cite{Falk1973}、文\cite{Vaish1976The}、文\cite{Sherali1980}的方法保证对所有情形都有限终止. 这其中, 文\cite{Falk1973}的方法需要每个父节点有$s+t-1$个后代, 这里$s$为$y$的维数, $t$为定义$\mathcal{X}$多面体的矩阵的行数. 但是子代问题的规模沿着分支会有效地减小; 文\cite{Vaish1976The}没有报告实施的情况, 但承认他们的程序需要大量的存储和列表处理; 而从报告的计算结果上看, 文\cite{Sherali1980}的程序似乎是最高效的. 这些方法中, 割平面法的收敛速度较慢. 这是因为当连续割平面接近最优解时, 平面彼此趋于平行, 这导致在每一次添加割平面时只切掉了可行域的一小部分. 而分支定界法在验证一个可行解是全局最优解时需要消耗大量的时间. 结合这两者的方法则可能会面临相同的问题. 
\par 问题\eqref{original problem element form}还可看做是一个非凸二次规划问题. 因此用于求解二次规划的算法可以直接用于问题\eqref{original problem element form}, 例如逐步二次规划 (SQP)算法、积极集法 (Active-set Methods)和内点法 (Interior-point methods), 可见文\cite{Nocedal2006Numerical}. 
\par 以上方法均以列向量作为求解对象. 直接用于如问题\eqref{original problem element form}以矩阵为变量的双线性问题将会引发计算量的问题.
\par 关于问题\eqref{bilinear programming}已有相当多的研究工作, 其中部分原因是其具有广泛的应用, 包括双矩阵博弈、动态马尔科夫分配问题、多商品网络流问题、某些动态生产问题、二次凹极小化问题、直线距离选址问题、三维分配问题以及一些互补规划问题. 可见文\cite{Vaish1974Nonconvex,Frieze1974A,Ibaraki1971Complementary,Konno1976Maximization,Konno1971Bilinear,Sherali1977The}.

\subsection{本文的工作}
我们充分考察了问题\eqref{original problem element form}的内部结构, 认识到其变量为矩阵以及其为双凸问题的本质, 采用交替方向乘子法 (Alternating Direction Methods of Multipliers, ADMM). 我们详细讨论了算法中子问题的求解, 对子问题设计了高效的求解算法. 最后, 我们给出了相关收敛性定理和数值实验, 比较了我们设计的算法与求解非凸二次规划的SQP和积极集法, 说明了我们所设计算法在求解大型问题时的优越性. 
\subsection{记号说明}
以下, 如不特别说明, $\Vert\cdot\Vert$表示欧式范数. $I$表示单位阵, 有时我们会加下标以表明它的阶数. $\trace(\cdot)$计算矩阵的迹, 或等价地, 矩阵特征值的和. 我们给矩阵加上标$T$表示转置运算. 我们用``$\circ$''表示两矩阵的阿达玛积, 用``$\otimes$''表示两矩阵的克罗内克积. 我们以符号``$\ge$''表示两矩阵间元素级的比较. 比如$A\ge B\Leftrightarrow a_{ij}\ge b_{ij},\forall i,j$. 一般地, 我们以$\mcL_0,\mcL_A$分别表示拉格朗日函数和增广拉格朗日函数. 有时我们会给它们加上标表明其归属或相关性. 我们以$\vectorize(\cdot)$表示矩阵的向量化, 以$\one$表示全1向量. 我们也使用$\langle\cdot,\cdot\rangle$表示$\mathbb{R}^n$中两向量的欧式内积. 在提及向量空间时, 我们以$N(\cdot)$表示线性算子的核空间, 以$\dim$表示空间的维数. 在算法描述中, 我们一般使用上标表示迭代数, 用下标表示分量.
\subsection{文章结构}
本文剩余部分组织如下: 我们在第\ref{preliminaries}节提供本文所需的预备知识. 我们在第\ref{optimality condition}节给出问题\eqref{original problem element form}的最优性条件, 并在第\ref{algorithm}节利用ADMM算法求解之. 第\ref{convergence analysis}节讨论算法的收敛性质. 第\ref{numerical experiments}节给出一些数值实验和讨论. 总结和未来工作展望可见第\ref{conclusion}节.