\section{预备知识}\label{preliminaries}
\subsection{标准矩阵内积}
为了更加简洁地叙述问题, 我们引入$\mathbb{R}^{n\times n}$中的一种特殊内积, 它由矩阵Frobenius范数直接诱导: $\forall A,B\in\mathbb{R}^{n\times n}$,
	$$\langle A,B\rangle:=\trace(A^TB)=\sum\limits_{i,j}a_{ij}b_{ij}.$$
	特别地, 
	$$\langle A,A\rangle=\trace(A^TA)=\Vert A\Vert_F^2.$$
	此内积对矩阵的求导规则为
	$$\frac{\mathrm d}{\mathrm dA}\langle A,B\rangle=B,\quad\frac{\mathrm d}{\mathrm dB}\langle A,B\rangle=A.$$
	因此, 上述问题\eqref{original problem element form}经过简单运算即可表述成简洁的矩阵形式:
	\begin{equation}
	\begin{array}{rl}
	\min\limits_{X,Y} & \langle R,X\rangle+\langle R, Y\rangle+\langle Y,XR\rangle\\
	\st & X\one=\rho,X^T\one=\rho,\trace(X)=0,X\ge0,\\
	& Y\one=\rho,Y^T\one=\rho,\trace(Y)=0,Y\ge0,
	\end{array}
	\label{original problem matrix form 1}
	\end{equation}
	其中$\one$为全1向量. 
	\begin{rem}
		考虑$X\ge0$的约束, $\trace(X)=0$就等价于$x_{ii}=0,\,i=1,2,\ldots,n$. 这对$Y$也是同样. 
	\end{rem}
	使用MATLAB求解二次规划的内置函数\texttt{quadprog()}, 我们在一些随机生成的问题上进行了实验, 发现所有输出的解$(\bar X,\bar Y)$均满足$\bar X=\bar Y$. 因此, 我们做出如下假设:
	\begin{assume}\label{assume}
		问题\eqref{original problem matrix form 1}的所有稳定点$(X,Y)$均满足$X=Y$.
	\end{assume}
	这样我们只需考虑下面的简化问题.
	\begin{equation}
		\begin{array}{rl}
			\min\limits_{X} & 2\langle X,R\rangle+\langle X,XR\rangle\\
			\st & X\one=\rho,X^T\one=\rho,\trace(X)=0,X\ge0.
		\end{array}
		\label{original problem matrix form 2}
	\end{equation}
	为方便后续的算法设计, 我们引入分裂变量$Z\in\mathbb{R}^{n\times n}$将问题的约束分成两部分, 同时将问题\eqref{original problem matrix form 2}目标函数的二次项替换成双线性项. 这样我们得到如下的等价问题:
	\begin{equation}
		\begin{array}{rl}
			\min\limits_{X,Z} & f(X,Z)\triangleq2\langle X,R\rangle+\langle Z,XR\rangle\\
			\st & X\one=\rho,\trace(X)=0,\\
			& Z^T\one=\rho,Z\ge0,\\
			& X=Z.
		\end{array}
		\label{original problem matrix form}
	\end{equation}
	
\subsection{凸集}
令$V$为实数域上的向量空间. 我们称$V$中的集合$\mathcal{C}$是凸的, 若对$\forall x,y\in\mathcal{C}\,,\forall t\in(0,1)$, 我们有
	\begin{equation*}
	(1-t)x+ty\in\mathcal{C}.
	%\label{definition of convex sets}
	\end{equation*}
	换句话说, 连接$x,y$的线段位于$\mathcal{C}$中. 由定义, 问题\eqref{original problem matrix form 1}的可行域就是个$\mathbb{R}^{n\times n}$中的凸集.