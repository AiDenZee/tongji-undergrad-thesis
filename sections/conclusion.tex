\section{总结与未来工作展望}\label{conclusion}
本文针对双线性规划问题\eqref{original problem element form}, 充分考虑其特殊结构, 设计了相应的ADMM算法. 之后我们证明了在迭代序列收敛的前提下, 算法收敛到问题\eqref{original problem matrix form}的稳定点. 最后, 我们在多个随机生成的问题上进行实验, 验证了算法的有效性, 并讨论了多个人工给定因子 (惩罚因子$\beta$、松弛因子$\alpha$)对算法效果的影响. 我们还将设计的算法与直接求解非凸二次规划的SQP算法和积极集法进行了比较, 得出了我们的算法在大型问题上更具优势的结论. 
\par 未来我们的工作将集中在以下几点:
\begin{enumerate}
\item 证明假设\ref{assume}, 即问题\eqref{original problem matrix form 1}与问题\eqref{original problem matrix form}的等价性. 
\item 设计求解$Z$子问题\eqref{Z subproblem}更加高效、精确的算法.
\item 证明算法\ref{BCD-ADMM}在较弱条件下 (比如无需迭代序列收敛的前提)的收敛性.
\item 对松弛因子、惩罚因子的进一步讨论. 我们在第\ref{numerical experiments}节中讨论了提前固定不同的松弛因子、惩罚因子对算法的影响. 但设计一些自适应调节松弛因子、惩罚因子的策略也是极有意义的.
\item 初始点对算法结果的影响. 显然, 问题\eqref{original problem element form}是非凸的二次规划, 因此必定有众多局部极小点. 尽管对于一般问题设计全局优化的算法并不可行, 但对于特殊的问题, 我们可以去证明算法的确能收敛到全局解. 影响求解非凸问题算法的一个重要因素就是初始点的选取.
\item 与现有双线性规划算法进行比较. 我们在第\ref{introduction}节中对现有的双线性规划算法做了简单介绍, 但我们在实验 (见第\ref{numerical experiments}节)中仅拿我们的算法与求解非凸二次规划的SQP算法和积极集法进行了比较. 相信更加丰富的实验比较能进一步说明我们所设计的算法的优越性. 同时, 我们也可以对比多个评价算法的指标, 除了运行时间, 还有计算复杂度等. 
\item 子问题的并行求解. 在第\ref{Z}节中我们设计了针对$Z$子问题的算法, 高效地求解了子问题. 在更新$Z$的过程中, 列与列之间的计算是互不影响的. 利用并行计算, 我们能期望在求解时间上获得较大的改善.
\end{enumerate}