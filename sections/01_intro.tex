\section{介绍}\label{sec:introduction}

在本节(\cref{sec:introduction})中,我们将讨论文档中常用的各种标题级别和字体样式。在文章中,标题是非常重要的组成部分之一,它可以帮助读者更好地理解文章的结构和内容。

\subsection{二级标题}

二级标题是一种较为重要的标题级别,一般用于表示文章中的主要章节或主题。通常,它们会在上面添加分割线或加粗等效果,以突出其重要性。

\subsubsection{三级标题}

相对于二级标题而言,三级标题是更加具体的标题级别,通常用于表示二级标题下的具体内容描述。它们的长度通常比二级标题短,与二级标题之间应有一定的间距。

\paragraph{段落标题}

段落标题是文章中比正文稍微具有一些重要性和突出性的内容,通常用加粗或斜体等方式来区别于正文。

\subparagraph{子段落标题}

子段落标题是相对于段落标题更加细节化的内容,用于突出一段文字中的重点内容。通常采用斜体或加粗的方式表示。在一些正式的文献中,子段落标题的使用较少。

\subsection{字体}

在下面的段落中,我们使用了不同的字体来表示不同的文字信息。下面是各段落所使用的字体和对应的命令:

\begin{itemize}
\item {\songti 宋体}:使用命令 \texttt{\textbackslash songti}。
\item {\heiti 黑体}:使用命令 \texttt{\textbackslash heiti}。
\item {\fangsong 仿宋}:使用命令 \texttt{\textbackslash fangsong}。
\item {\kaishu 楷书}:使用命令 \texttt{\textbackslash kaishu}。
\end{itemize}

{\songti 

1900年前后,由埃里希·宝隆创办的“同济医院”正式挂牌。医院的医师大都是“德医公会”成员。他们白天忙于经营自己的诊所,只有傍晚到医院看门诊、动手术。埃里希·宝隆医生看到医院里的医疗力量不足,计划在院内设立一所德文医学堂,招收中国学生,以培养施诊医生。这个计划得到德国驻沪总领事以及德国政府高等教育司的支持。1906年,他们设立了一个支持医学堂开办的基金会,得到了德国“促进德国与外国思想交流的科佩尔基金会”的协助,筹集到一批医科书刊及新式的外科手术电动器械等物品。}

{\heiti 1907年6月医学堂开学前,德国驻沪总领事克纳佩在上海不仅号召德国商人捐款,而且要求德国洋行向中国商人募捐。同时,费舍尔还要求中国官方的资助和支持,克纳佩利用在中德两国募来的捐款,成立了“为中国人办的德国医学堂基金会”。当时规定,捐款金额较多者可成为医学堂董事会董事。医学堂建立时定名为德文医学堂,并成立了董事会负责学校的管理。董事会由18人组成,主要成员有:三个德医公会元老:宝隆、福沙伯(第二任校长)、福尔克尔;三名德国商人:莱姆克、米歇劳和赖纳;两名中国绅商:朱葆三(沪军都督府财政部长及上海商务会会长,大买办)、虞洽卿(荷兰银行买办);总领事馆的副领事弗赖海尔·冯·吕特等。埃里希·宝隆医生被正式推选为董事会总监督(董事长)兼学堂首任总理(校长),负责学堂的管理。医学堂的校址设在同济医院对面的白克路(今凤阳路415号上海长征医院内)。1907年10月1日德文医学堂举行了开学典礼。}

{\ifcsname fangsong\endcsname\fangsong\else[无 \cs{fangsong} 字体。]\fi 
1923年3月17日北洋政府教育部下达第108号训令,批准同济工科“改为大学”。学校随即召开董事会议,将学校定名为“同济大学”。1923年3月26日,学校以“同济大学董事会”名义呈文北洋政府教育部,称“经校董会议定名称为同济大学”。1923年4月24日,北洋政府教育部下达第634号“指令”,称“该校名称拟改为同济大学,应予照准备案”。1924年5月20日北洋政府教育部下达第120号训令,批准同济医科为大学。从此以后,5月20日定为校庆日。}

{\ifcsname kaishu\endcsname\kaishu\else[无 \cs{kaishu} 字体。]\fi 
抗日战争胜利后,1946年,国立同济大学分批迁回上海。由于缺少校舍,学校分散教学,成为斜跨上海市区、分散十多处的“大学校”。其中学校办公室和医学院位于善钟路100弄10号(今常熟路),附属医院分别为白克路上的中美医院(原宝隆医院,今凤阳路)和同孚路82号(今石门一路)的原德国医院,理学院在平昌街日本第七国民学校内(今国顺路上海电视大学),文法学院在四川北路(今复兴初级中学),新生院位于江湾新市区的市图书馆(今黑山路),高级工业职业学校则在江湾魏德迈路(今邯郸路),附属中学位于市博物馆(今长海医院飞机楼)。工学院位于其美路的原日本中学(今杨浦区四平路1239号),1949年后逐步发展成同济大学的主校区。}

\subsubsection{测试生僻字}\label{sec:uncommon}

本小节对生僻字\footnote{此处的生僻字指:GBK编码中有,但GB2312编码中没有的字。}的显示进行测试。

丂丄丅丆丏丒丗丟丠両丣並丩丮丯丱丳丵丷丼乀乁乂乄乆乊乑乕乗乚乛乢乣乤乥乧乨乪乫乬乭乮乯乲乴乵乶乷乸乹乺乻乼乽乿亀亁亃亄亅亇亊亐亖亗亙亜亝亣亪亯亰亱亴亶亷亸亹亼亽亾仈仌仏仐仒仚仛仜仠仢仦仧仩仭仮仯仱仴仸仹仺仼仾伀伂伃伄伅伆伇伈伋伌伒伓伔伕伖伜伝伡伣伨伩伬伭伮伱伳伵伷伹伻伾伿佀佁佂佄佅佇佈佉佊佋佌佒佔佖佡佢佦佨佪佫佭佮佱佲併佷佸佹佺佽侀侁侂侅侇侊侌侎侐侒侓侕侘侙侚侜侞侟価侢侤侫侭侰侱侲侳侴侶侷侸侹侺侻侼侽侾俀俁係俆俇俈俉俋俌俍俒俓俔俕俖俙俛俢俤俥俧俫俬俰俲俴俵俶俷俹俻俼俽俿倀倁倂倃倄倅倇倈倊倎倐倓倕倖倗倛倝倞倠倢倣値倧倯倰倱倲倳倴倵倶倷倸倹倻倽倿偀偁偂偄偅偆偊偋偍偐偑偒偓偔偖偗偘偙偛偝偞偟偠偡偢偣偤偦偧偨偩偪偫偭偮偯偰偱偲偳偸偹偺偼偽傁傂傃傄傆傇傉傊傋傌傎傏傐傑傒傓傔傕傖傗傚傛傜傝傞傟傠傡傢傤傦傪傫傮傯傰傱傴傶傸傹傼傽傿僀僁僂僃僄僆僇僈僉僊僋僌働僎僐僒僓僔僗僘僙僛僜僝僟僠僡僢僣僤僨僩僪僫僯僰僱僲僴僶僷僸僺僼僽僾僿儁儂儃儅儊儌儍儎儏儐儑儓儔儕儖儗儘儙儚儛儜儝儞儠儢儣儤儥儦儧儨儩儫儬儭儮儯儰儱儳儴儵儶儷儸儹儺儻儼儽。