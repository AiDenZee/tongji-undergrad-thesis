\section{简介}\label{sec:introduction}

在\cref{sec:introduction}中,我们介绍了文档中常用的各级标题和正文等样式。其中,二级标题是用来表示一级标题下的重要内容,而三级标题则是更加具体的内容描述。举例来说,在同济大学的介绍中,二级标题介绍了该校的基本情况,而三级标题(\cref{subsec:third-level-title})则介绍了该校的历史沿革。

% 同时,我们还可以使用标签和引用功能,在需要的地方引用对应的小节,如“在\cref{subsec:third-level-title}中介绍了同济大学的历史沿革”。另外,本节还提到了脚注的使用方式。

\subsection{二级标题}

同济大学(英语:Tongji University),简称同济,是位于中国上海市的一所综合性大学,是中华人民共和国教育部直属的全国重点大学,行政级别为副部级,是“双一流A类”和原“985工程”、原“211工程”重点建设大学,同时是卓越大学联盟、全球环境与可持续发展合作联盟(GUPES)、国际设计艺术院校联盟(Cumulus)、21世纪学术联盟(AC21)、同济-伯克利工程联盟(Tongji-Berkeley Alliance)、中俄工科大学联盟(ASRTU)成员。

\subsubsection{三级标题}\label{subsec:third-level-title}

同济大学的前身是1907年创办的德文医学堂,后改名为同济德文医学堂;1912年与创办不久的同济德文工学堂合并,更名为同济德文医工学堂;1923年正式定名为同济大学;1927年成立国立同济大学,是中国最早的七所国立大学之一;1949年更名为同济大学。


\paragraph{段落标题}

同济大学现有16个学院,覆盖了工、理、管、文、法、医、艺术、建筑等多个学科领域。学校有来自全国各地、国际各国的学生和教师,是一所具有世界影响力的综合性大学。

\subparagraph{子段落标题}

同济大学在教育和科学研究方面具有较高的声誉。学校在工程、建筑、交通、材料科学、环境科学、医学、文化与创意产业等领域都有较高的学科实力和影响力。同时,学校重视国际合作与交流,与多所国际知名大学建立了紧密的合作关系。

\subsection{字体}

在下面的段落中,我们使用了不同的字体来表示不同的文字信息。下面是各段落所使用的字体和对应的命令:

\begin{itemize}
\item {\songti 宋体}:使用命令 \texttt{\textbackslash songti}。
\item {\heiti 黑体}:使用命令 \texttt{\textbackslash heiti}。
\item {\fangsong 仿宋}:使用命令 \texttt{\textbackslash fangsong}。
\item {\kaishu 楷书}:使用命令 \texttt{\textbackslash kaishu}。
\end{itemize}

{\songti 

1900年前后,由埃里希·宝隆创办的“同济医院”正式挂牌。医院的医师大都是“德医公会”成员。他们白天忙于经营自己的诊所,只有傍晚到医院看门诊、动手术。埃里希·宝隆医生看到医院里的医疗力量不足,计划在院内设立一所德文医学堂,招收中国学生,以培养施诊医生。这个计划得到德国驻沪总领事以及德国政府高等教育司的支持。1906年,他们设立了一个支持医学堂开办的基金会,得到了德国“促进德国与外国思想交流的科佩尔基金会”的协助,筹集到一批医科书刊及新式的外科手术电动器械等物品。}

{\heiti 1907年6月医学堂开学前,德国驻沪总领事克纳佩在上海不仅号召德国商人捐款,而且要求德国洋行向中国商人募捐。同时,费舍尔还要求中国官方的资助和支持,克纳佩利用在中德两国募来的捐款,成立了“为中国人办的德国医学堂基金会”。当时规定,捐款金额较多者可成为医学堂董事会董事。医学堂建立时定名为德文医学堂,并成立了董事会负责学校的管理。董事会由18人组成,主要成员有:三个德医公会元老:宝隆、福沙伯(第二任校长)、福尔克尔;三名德国商人:莱姆克、米歇劳和赖纳;两名中国绅商:朱葆三(沪军都督府财政部长及上海商务会会长,大买办)、虞洽卿(荷兰银行买办);总领事馆的副领事弗赖海尔·冯·吕特等。埃里希·宝隆医生被正式推选为董事会总监督(董事长)兼学堂首任总理(校长),负责学堂的管理。医学堂的校址设在同济医院对面的白克路(今凤阳路415号上海长征医院内)。1907年10月1日德文医学堂举行了开学典礼。}

{\ifcsname fangsong\endcsname\fangsong\else[无 \cs{fangsong} 字体。]\fi 
1923年3月17日北洋政府教育部下达第108号训令,批准同济工科“改为大学”。学校随即召开董事会议,将学校定名为“同济大学”。1923年3月26日,学校以“同济大学董事会”名义呈文北洋政府教育部,称“经校董会议定名称为同济大学”。1923年4月24日,北洋政府教育部下达第634号“指令”,称“该校名称拟改为同济大学,应予照准备案”。1924年5月20日北洋政府教育部下达第120号训令,批准同济医科为大学。从此以后,5月20日定为校庆日。}

{\ifcsname kaishu\endcsname\kaishu\else[无 \cs{kaishu} 字体。]\fi 抗日战争胜利后,1946年,国立同济大学分批迁回上海。由于缺少校舍,学校分散教学,成为斜跨上海市区、分散十多处的“大学校”。其中学校办公室和医学院位于善钟路100弄10号(今常熟路),附属医院分别为白克路上的中美医院(原宝隆医院,今凤阳路)和同孚路82号(今石门一路)的原德国医院,理学院在平昌街日本第七国民学校内(今国顺路上海电视大学),文法学院在四川北路(今复兴初级中学),新生院位于江湾新市区的市图书馆(今黑山路),高级工业职业学校则在江湾魏德迈路(今邯郸路),附属中学位于市博物馆(今长海医院飞机楼)。工学院位于其美路的原日本中学(今杨浦区四平路1239号),1949年后逐步发展成同济大学的主校区。}


