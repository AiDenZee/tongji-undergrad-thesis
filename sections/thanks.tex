\section*{谢辞}
\addcontentsline{toc}{section}{谢辞}
%章节不标号时用\section*{},并用\addcontentsline{toc}{section}{}将其加入目录
谨在论文完成之际, 我要把我最真挚的谢意献给我的两位论文指导老师——我院殷俊锋教授和中国科学院数学与系统科学研究院的刘歆副研究员. 殷老师是我在同济大学数学科学学院的数值分析和计算方法任课老师, 是将我引入计算数学大门的人. 您渊博的学识在课堂上展现得淋漓尽致, 您独特的幽默感总能给课程注入活力. 您从不将学生的视野局限在课本知识, 反倒是推动我们自发地在学术的田野中开疆扩土. 在您的课堂里, 我们能了解到学术的最新动态, 感受到学术的魅力. 与我而言, 您不仅是一位严师, 更像是一位慈父. 您从不吝惜课堂以外的时间, 乐意与学生交流. 我也从您的口中学到了许多学术以外的道理. 您曾经向我解说国内外的经济局势, 提醒我要不时关注国家大事; 您曾经私信我长文, 为我讲述做人做事做学术的道理. 在我的心中, 您的形象早已超越了老师的存在. 刘老师是我未来五年在中科院的指导老师. 我从您的身上看到了学者的谦逊、师者的威严、父亲的慈爱、丈夫的体贴和跑者的坚守. 在访问您的短短两个多月时间我学到了太多. 您教会我如何写作汇报和论文、如何做一位合格的学生、如何遵守学术的严谨、如何妥善利用身边的资源以及如何保持一颗积极向上的心. 与您的每次交谈, 我都能收获受益终生的警句和观点. 您始终是一个给予的角色, 一心一意为学生构建美好的未来. 在您的帮助下, 我有幸作为志愿者参加了丝路数学中心的国际会议, 与国际优化领域的专家面对面接触, 获得书本之外的宝贵经历. 在与外国学者专家的长期交流后, 我的英语口语和写作能力均获得了极大的提升. 同时您也带我加入了您领衔的优化跑马组, 让我享受丰富的经验和咨询, 打造更健康的身体. 您给了我无数次机遇, 我必不负所期.
\par 我还要特别感谢我院副院长潘生亮教授. 您是帮助我打下分析基础的恩师, 平常课上课下教会了我许多的道理. 同时我还要感谢我院院长许学军教授. 若没有您和潘老师, 我也不会有未来的中科院之旅. 我也感谢您的谆谆教诲. 记得去中科院面试时, 您提醒我不要忘记自己是同济数院人, 在外要有担当, 刻苦学习, 对内要懂得回馈母校, 永怀感恩之心. 
\par 我要感谢蒋志洪教授、陈滨副教授、周羚君副教授以及其他悉心教导过我的任课老师. 感谢你们长期以来的鼓励和帮助. 感谢(曾经)团委和负责学生工作的李扬帆老师、郁霞老师、赵盈老师和金文心老师, 我的三位班主任——王鹏教授(现在福建师范大学)、彭婧老师、尚培培副教授. 大学四年给你们添了不少麻烦, 而你们总是满怀热情地帮助我解决问题. 其中特别感谢李扬帆老师和彭婧老师. 大一一进数院我们就称呼李老师``帆姐''. 在我的印象里``帆姐''永远焦头烂额地忙于各种学生工作. 这四年来您辛苦了! 您永远是我们的``帆姐''! 因为岁数差得不大, 我总是称呼彭婧老师为``婧哥哥''. 你是我在``数学外卖''中的引路人, 不论是组员时还是组长时. 也多亏了你, 我总是敢于挑战一些自己没有做过的事. 从你的身上, 我学到了对待他人要打心底里热心, 要善于伸出援助之手. 你说我们这些人毕业了以后你会哭, 其实我们也不例外. 这样的情感是世间少有的.
\par 我要感谢曾经帮助过我的王锦东、姚天洋、郝亦雯、吴悦同、王铭恺等同学, 我的室友、学弟学妹们. 你们为我原本枯燥的大学生活增添了色彩. 我要特别感谢我的两位好友——刘炳言和宋增春. 以前我们三人常年``混迹''于各种讨论班, 课上课下讨论问题, 分享自己的生活和爱好. 尽管今朝各奔东西, 但今后我们也是永远的好朋友. 
\par 我要感谢曾经和我一同参加比赛的张越、周柳曼、汤晨宇、张远航、王子鑫等同学. 感谢我们一起奋斗过的时光. 其中尤其要感谢我的死党张越. 尽管你是我的高中(不同班的)同学, 但我大学才认识你. 相见恨晚, 我们有许多共同的爱好, 曾经互相倾诉过自己的不快. 你拥有积极向上的生活态度和一颗温暖待人的心. 期待你在未来闯出自己的天地.
\par 我要感谢在我访问中科院计算数学所时帮助过我的各位老师、师兄师姐和同学们. 感谢你们带我融入新的环境, 让我得以快速地进入状态. 其中尤其要感谢刘颖老师、刘为师兄、陈雅丹师姐、肖纳川师兄、高斌师兄、谢鹏程、裴骞、郭仲琨.
\par 最后, 我要感谢我的家人和女朋友. 是你们无微不至的关怀和无私的爱给予了继续向前的勇气, 使我在自己选择的道路上不畏挫折, 坚定地走下去. 我会满怀热情地面对今后的学习、工作以回报他们的殷切希望.