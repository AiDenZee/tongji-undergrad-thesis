\begin{center}\bf\heiti\zihao{2}
	\ \\
	求解一类双线性规划问题的数值算法
\end{center}
\vspace{1pt}
\begin{center}\heiti\zihao{4}
	摘要
\end{center}
\hspace{2mm}
{\heiti\zihao{-5}}\quad {\songti\zihao{-5}我们考虑一类特殊的双线性规划问题, 其中变量为矩阵, 目标函数为双线性函数, 约束有单纯形约束、非负约束以及对角元为0的约束. 该问题可来源于材料计算中的最优运输问题. 我们分别从双线性规划和非凸二次规划两个方面回顾了现有的求解方法, 并指出了直接应用它们的弊端. 基于对问题特殊结构的剖析, 我们提出了基于交替方向乘子法的算法, 详细地介绍了其中子问题的求解. 之后, 我们给出了一些收敛性分析. 我们证明了在一定条件下我们的算法产生的迭代序列收敛于问题的稳定点. 最后我们在随机生成的小型、大型问题上进行数值实验, 讨论了算法中不同松弛因子、惩罚因子对算法效果的影响. 我们还比较了本文算法与求解非凸二次规划的逐步二次规划算法、积极集法, 得出本文算法在大型问题上更具优势的结论. 我们在文章的末尾给出了总结和展望.
}
%局部修改字体字号{\字体名\zihao{} ...}
\par
\vskip 5mm
\noindent {\heiti\zihao{5}关键词:}\quad{\songti\zihao{-5}  双线性规划, \quad 非凸二次规划, \quad 交替方向乘子法, \quad 子问题, \quad 收敛性证明}~\\


\newpage
% \vskip 4mm
\begin{center}\heiti\zihao{2}
	{\bf {Numerical Algorithms for a Class of Bilinear Programming} }
\end{center}
\vspace{1pt}
\begin{center}\heiti\zihao{4}
	\bf{	ABSTRACT}
\end{center}
\hspace{2mm}
{\heiti\zihao{-5}}\quad {\songti\zihao{-5}We consider a special class of bilinear programming problems, in which the variables are matrices, the objective is bilinear and the constraints include simplex constraints, nonnegative constraints, and zero-diagonal-element constraints. The problem can be derived from the optimal transport problem in material computation. We review the existing methods from the points of both bilinear programming and nonconvex quadratic programming, and point out the drawbacks of directly applying them to the problem. By exploiting the special structure of the problem, we propose a algorithm based on Alternating Directions Methods of Multipliers. We also describe how to solve subproblems in detail. After that, we provide some convergence analysis. We prove that the iterative sequence generated by our algorithm converges to the critical point of the problem under certain conditions. Finally, we carry out the algorithm on some randomly generated small and large problems. Discussions on the effect of different relaxation factors and penalty factors are included. Furthermore, we make a comparison among our algorithm, Sequential Quadratic Programming, and Active-Set Methods. Our algorithm turns out to outperform the latter two methods on large problems. Conclusion and prospects can be viewed at the end of this paper.
}\par


\vskip 4mm 
\noindent{\heiti\zihao{5}\bf {Keywords:}}\quad
{\songti\zihao{-5} Bilinear Programming, \quad Nonconvex Quadratic Programming, \quad Alternating Direction Methods of Multipliers, \quad Subproblem, \quad Convergence Analysis}
\par